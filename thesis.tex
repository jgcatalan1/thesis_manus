\documentclass[12pt,letterpaper]{report}

% Required packages
\usepackage[utf8]{inputenc}
\usepackage[T1]{fontenc}
\usepackage{amsmath,amssymb,amsthm}
\usepackage{graphicx}
\usepackage[margin=1in]{geometry}
\usepackage{setspace}
\usepackage{titlesec}
\usepackage{caption}
\usepackage{hyperref}

% Custom style file for thesis formatting requirements
% thesis-style.tex
% Custom style file for thesis manuscript formatting requirements

% ============================================================================
% HEADING FORMATTING
% ============================================================================

% 1st Level: ALL CAPS, CENTERED (chapters)
\titleformat{\chapter}[display]
  {\normalfont\bfseries\centering}
  {\MakeUppercase{\chaptertitlename\ \thechapter}}
  {20pt}
  {\MakeUppercase}

% 2nd Level: Title Case (sections)
\titleformat{\section}
  {\normalfont\bfseries}
  {\thesection}
  {1em}
  {}

% 3rd Level: Sentence case (subsections)
\titleformat{\subsection}
  {\normalfont\bfseries}
  {\thesubsection}
  {1em}
  {}

% 4th Level: Sentence case, run-in with paragraph (subsubsections)
\titleformat{\subsubsection}[runin]
  {\normalfont\bfseries}
  {\thesubsubsection}
  {1em}
  {}
  [.]

% ============================================================================
% PROHIBIT BULLETS AND EM DASHES
% ============================================================================

% Redefine itemize to produce error
\renewenvironment{itemize}{%
  \PackageError{thesis-style}{Bullet points are prohibited in this thesis manuscript}{%
    Please use numbered lists or paragraph text instead of bullet points.}%
}{}

% Redefine enumerate to allow numbered lists (not bullets)
% Keep enumerate functional for academic writing

% Create a simple warning for em dashes instead of preventing them
% This allows the document to compile but reminds authors about the restriction
\newcommand{\notemdash}{%
  \PackageWarning{thesis-style}{Em dashes should be avoided in this thesis manuscript}%
}

% ============================================================================
% FIGURE AND TABLE GUIDELINES
% ============================================================================

% Configure captions to appear below figures and tables
\captionsetup{position=below}

% Add a command to remind about proper figure/table introduction
\newcommand{\figurereminder}{%
  \PackageWarning{thesis-style}{%
    Remember: Introduce figures in the preceding paragraph, 
    place caption below the figure, and discuss in the subsequent paragraph}%
}

\newcommand{\tablereminder}{%
  \PackageWarning{thesis-style}{%
    Remember: Introduce tables in the preceding paragraph,
    place caption below the table, and discuss in the subsequent paragraph}%
}

% ============================================================================
% MATHEMATICAL FORMULAS
% ============================================================================

% Mathematical packages are already loaded in main document
% Ensure proper rendering with amsmath, amssymb, amsthm

% ============================================================================
% ACADEMIC STYLE ENFORCEMENT
% ============================================================================

% Ensure double spacing for academic manuscripts
\doublespacing

% Set proper academic formatting for bibliography
\bibliographystyle{plain}


\begin{document}

\title{Thesis Title}
\author{Student Name}
\date{\today}
\maketitle

\doublespacing

\chapter*{ABSTRACT}
\addcontentsline{toc}{chapter}{ABSTRACT}

This is the abstract of the thesis. It provides a brief overview of the research conducted.

\tableofcontents
\listoffigures
\listoftables

\chapter{INTRODUCTION}

This is the introduction chapter. It sets up the context for the research.

\section{Background and Motivation}

This section provides background information.

\subsection{Related work}

This subsection discusses related work in the field.

\subsubsection{Specific topic.}
This is a run-in heading that continues with the paragraph text immediately following the heading.

\section{Research Questions}

This section outlines the research questions addressed in this thesis.

\chapter{METHODOLOGY}

This chapter describes the methodology used in the research.

\section{Experimental Setup}

The experimental setup is described in this section. In this work, we utilize mathematical formulas to express key relationships. For instance, the quadratic formula is given by:
\begin{equation}
x = \frac{-b \pm \sqrt{b^2 - 4ac}}{2a}
\end{equation}
where $a$, $b$, and $c$ are coefficients of the quadratic equation $ax^2 + bx + c = 0$.

\section{Example with Figure}

This section demonstrates the proper way to introduce and discuss figures. The following paragraph introduces the figure.

Figure \ref{fig:example} presents an example visualization of the data collected during the experimental phase. The figure illustrates the key trends observed in the dataset.

\begin{figure}[htbp]
\centering
% \includegraphics[width=0.8\textwidth]{example-figure}
\fbox{\parbox{0.8\textwidth}{\centering [Example Figure Placeholder]}}
\caption{Example figure showing experimental results}
\label{fig:example}
\end{figure}

The figure reveals several important patterns in the data. First, we observe a linear relationship between the two variables. Second, the variance increases with higher values of the independent variable. These observations support the hypotheses presented in the previous chapter.

\section{Example with Table}

This section demonstrates the proper way to introduce and discuss tables. The following paragraph introduces the table.

Table \ref{tab:example} summarizes the key statistics from the experimental results. The table presents mean values and standard deviations for each experimental condition.

\begin{table}[htbp]
\centering
\caption{Summary statistics for experimental conditions}
\label{tab:example}
\begin{tabular}{lcc}
\hline
Condition & Mean & Std Dev \\
\hline
Control & 10.5 & 2.3 \\
Treatment A & 15.2 & 3.1 \\
Treatment B & 18.7 & 2.8 \\
\hline
\end{tabular}
\end{table}

The table shows that Treatment B achieved the highest mean value of 18.7, which represents a 78\% increase over the control condition. Treatment A also showed improvement with a mean of 15.2. The standard deviations remain relatively consistent across conditions, indicating stable measurement precision.

\chapter{RESULTS}

This chapter presents the results of the research.

\chapter{DISCUSSION}

This chapter discusses the implications of the results.

\chapter{CONCLUSION}

This chapter concludes the thesis and suggests directions for future work.

\bibliographystyle{plain}
\bibliography{references}

\end{document}
